The first of the two macros that \koDi\ offers is \lstinline|\obj|.
It is polymorphic and can draw both single objects and layouts.

\begin{lstlisting}
\obj@opt@ <object options> @/opt@{<math>};(@
  \marginpar{\scriptsize Orange denotes optional fragments.}@)
\obj@opt@ <layout options> @/opt@{<layout>};
\end{lstlisting}

Layouts are described using the customary \TeX\ tabular syntax.

\begin{lstlisting}
<layout>         := (@\itshape\underbar{<row> <row separator>}@)(@
  \marginpar{\scriptsize Underlined fragments are repeated one or more times.}@)
<row>            := <cell> (@\itshape\color{orange!80!black}\underbar{<cell separator> <cell>}@)
<row separator>  := \\ @opt@[<length>]@/opt@
<cell>           := @opt@|<object options>| @/opt@<math>
<cell separator> := & @opt@[<length>]@/opt@
\end{lstlisting}

The discretionary options syntax is analogous to standard \TikZ\ nodes and
matrices, respectively.

\begin{lstlisting}
<object options> := (@\itshape\color{orange!80!black}\underbar{[object keylist]}@) @opt@(<name>) at (<coordinate>)@/opt@
<layout options> := (@\itshape\color{orange!80!black}\underbar{[layout keylist]}@) @opt@(<name>) at (<coordinate>)@/opt@
\end{lstlisting}

\hfill$\therefore$\hfill\null

Very little of the given syntax is specific to \koDi.
\TikZ\ options are easy to pick up on the way,
blah blah blah.

Here is a kitchen sink for tabular syntax:

\begin{tcblisting}{kodi snippet}
\SmashAndCenter{\tikzpicture[kodi, /kD/diagrams/square=2em]
\obj { A & B &[1em] C \\
       D & E &      F \\[-1em]
       G & H &      I \\ };
\endtikzpicture}
\end{tcblisting}

Objects are automagically named; the latest homonymous prevails.

\begin{tcblisting}{kodi snippet}
\SmashAndCenter{\tikzpicture[kodi, /kD/diagrams/square=2.125em]
\obj { A & A \\ };
\draw (A) circle (1em);
\endtikzpicture}
\end{tcblisting}

Naming a specific object avoids its automatic labeling.

\begin{tcblisting}{kodi snippet}
\SmashAndCenter{\tikzpicture[kodi, /kD/diagrams/square=2.125em]
\obj { A & |(A')| A \\ };
\draw [red]   (A)  circle (1em);
\draw [green] (A') circle (1em);
\endtikzpicture}
\end{tcblisting}

Naming a layout lets you refer to its objects by row and column.

\begin{tcblisting}{kodi snippet}
\SmashAndCenter{\tikzpicture[kodi, /kD/diagrams/square=2.125em]
\obj (M) { A & A \\ A & A \\ };
\draw [red]   (M-1-2) circle (1em);
\draw [green] (M-2-1) circle (1em);
\endtikzpicture}
\end{tcblisting}