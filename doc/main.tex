% fewsh was here
\documentclass[12pt]{scrbook}
\usepackage[british]{babel}

% \hyphenation{Fortran hy-phen-ation}

\usepackage{libertine}
\usepackage{libertinust1math}
\usepackage[ttdefault=true]{AnonymousPro}

\usepackage[utf8]{inputenc}
\usepackage[T1]{fontenc}

% \def\id{{\mathbf 1}}
% \DeclareUnicodeCharacter{0131}{{\noexpand\mathbf R}}
% \DeclareUnicodeCharacter{00D7}{{\otimes}}
% \catcode`ı=\active \letı\id
% \catcode`×=\active \let×\otimes
% \usepackage{newunicodechar}
% \newunicodechar{ı}{\mathbf 1}
% \newunicodechar{×}{\otimes}

\setkomafont{disposition}{\rmfamily\scshape}

\usepackage{geometry}

\geometry{
  a4paper,
  portrait,
  marginparwidth=4.25cm,
  marginparsep=.75cm,
  % showframe,
  width=11cm,
  hmarginratio=10:25,
  vmarginratio=20:30,
%  footskip=.5in,
}
%\savegeometry{main}

%\usepackage{multicol}
%\usepackage{tabularx}
%\usepackage{verbatimbox}
% \usepackage{wrapfig}

\usepackage{parskip}

\usepackage{mathtools}
\usepackage{amsfonts}
\usepackage{amssymb}


\DeclareMathOperator{\coker}{coker}

\usepackage{float}
% \usepackage{framed}
% \usepackage{lipsum}
% \usepackage{enumitem}
% \usepackage{pdfpages}

% \usepackage{caption}
\usepackage{sidenotes}

% \DeclareCaptionStyle{sidecaption}%
% {font=footnotesize,labelfont=sc}

% \usepackage{csquotes}

\usepackage{graphicx}
% \usepackage[makeroom]{cancel}

\usepackage{etoolbox}

% \usepackage{todonotes}
% \usepackage{marginnote}

% \usepackage[lastpage,user]{zref}
% \usepackage[hidelinks]{hyperref}

\usepackage{xcolor}
\usepackage{tikz}
% \usetikzlibrary{cd}
% \usetikzlibrary{quotes}
% \pgfkeys{/handlers/first char syntax=false}

\usetikzlibrary{kodi}
\usetikzlibrary{positioning}

% \usepackage[all]{xy}
% \usepackage{pstricks,pst-node}


\usepackage{xparse}

\usepackage[
  headwidth=textwithmarginpar,
  footwidth=textwithmarginpar,
]{scrlayer-scrpage}
%\clearpairofpagestyles

% \lofoot{}
% \cofoot{}
% \rofoot{}
% \lohead{}
% \cohead{}
% \rohead{\thepage}

%\usepackage{multicol}

\usepackage{microtype}

\usepackage{listings}

% http://tex.stackexchange.com/a/336331/82186
\makeatletter
\lst@Key{lastline}\relax{\ifnumcomp{#1}{<}{0}{%
  \let\mylst@file\lst@intname\sbox0{\lstinputlisting{\mylst@file}}%
  \def\lst@lastline{\the\numexpr#1+\value{lstnumber}-1\relax}}%
  {\def\lst@lastline{#1\relax}}}
\makeatother



\usepackage{showexpl}
\makeatletter
\lst@Key{postset}\relax{\def\SX@postset{#1}}
\newcommand\SX@postset{}
\renewcommand*\SX@resultInput{%
  \ifx\SX@graphicname\@empty
    \begingroup
      \MakePercentComment\makeatother\catcode`\^^M=5\relax
      \SX@@preset\SX@preset
      \if@SX@rangeaccept
       \let\SX@tempa=\SX@input
      \else
       \let\SX@tempa=\input
      \fi
      \if\SX@scaled ?%
        \let\SX@tempb=\@firstofone
      \else
        \if\SX@scaled !%
          \def\SX@tempb##1{\resizebox{\SX@width}{!}{##1}}%
        \else
          \def\SX@tempb##1{\scalebox{\SX@scaled}{##1}}%
        \fi
      \fi
      \SX@tempb{\SX@tempa{\SX@codefile}}\SX@postset\par
    \endgroup
  \else
    \expandafter\includegraphics\expandafter[\SX@graphicparam]%
      {\SX@graphicname}%
  \fi
}
\makeatother

\lstdefinelanguage{TikZ}
{morekeywords={for},
sensitive=false,
morecomment=[l]{//},
morecomment=[s]{/*}{*/},
morestring=[b]",
}

\usepackage{color,soul}
\definecolor{darkblue}{rgb}{0,0,0.5}
\setulcolor{darkblue}

\lstset{
  language=[LaTeX]TeX,
  basicstyle=\ttfamily\lst@ifdisplaystyle\scriptsize\fi,
  backgroundcolor=\color{teal!5},
  % keywordstyle=*\color{blue},
  % identifierstyle=\color{orange}\bfseries,
  % morekeywords={\path},
  moretexcs={
    \starttext,\stoptext,\usemodule,
    \tikzpicture,\endtikzpicture,
    \tikzexternalize,
    \starttikzpicture,\stoptikzpicture,
    \usetikzlibrary,
    \kodi,\endkodi,
    \startkodi,\stopkodi,
    \lay,\obj,\mor,
    \bye
  },
  texcsstyle=*\textbf,
  % morestring=[b]",
  commentstyle=\itshape,
  frame=none,
  % extendedchars=false,
  % inputencoding=utf8,
  escapeinside={(@}{@)},
  moredelim=**[is][\color{orange!80!black}]{@opt@}{@/opt@},
  moredelim=**[s][\itshape]{<}{>},
  literate={:=}{{$\equiv$}}1 {~}{{\textvisiblespace}}1,
}

% \makeatletter
% \newcount\ublvl\ublvl=0
% \newcount\ubdpt\ubdpt=0
% \newdimen\ubgap\ubgap=.2em
% \def\underbra#1{\underline {\sbox \tw@ {\global\advance\ubdpt1\advance\ublvl1#1}\dp \tw@ \dimexpr\ubgap*(\ubdpt-\ublvl-1)\relax \box \tw@ }\ifnum\ublvl=0\ubdpt=0\fi}
% \makeatother

\lstset{explpreset={
  wide,
  basicstyle=\ttfamily\scriptsize,
  pos=o,
  width=\marginparwidth,
  hsep=\marginparsep,
  rframe={},
  preset={\centering\tikzpicture[kodi]},
  postset={\endtikzpicture}
}}

\usepackage{lipsum}
\usepackage{caption}
\usepackage{subcaption}

\captionsetup[marginfigure]{
  textfont=bf,
  justification=centering
}
\captionsetup[subfigure]{
  textfont=normalfont,
  singlelinecheck=off,
  justification=centering
}

\usepackage{hologo}
\def\ConTeXt{\hologo{ConTeXt}}
\def\koDi{{\scshape koDi}}
\def\TikZ{{\scshape TikZ}}

\usepackage{tcolorbox}
\tcbuselibrary{listingsutf8}
\tcbset{listing utf8=latin1}

\tcbset{kodi snippet/.style={
  size=tight,
  colback=white,
  colframe=white,
  if odd page={
    listing side text,
    lefthand width=\textwidth,
    righthand width=\marginparwidth,
    halign lower=center,
  }{
    text side listing,
    righthand width=\textwidth,
    lefthand width=\marginparwidth,
    halign upper=center,
  },
  toggle enlargement,
  grow to right by=\marginparsep+\marginparwidth,
  sidebyside gap=\marginparsep,
  listing options={
    firstline=2,
    lastline=-1
  }
}}

\tcbset{xy snippet/.style={
  size=tight,
  colback=white,
  colframe=white,
  if odd page={
    listing side text,
    lefthand width=\textwidth,
    righthand width=\marginparwidth,
    halign lower=center,
  }{
    text side listing,
    righthand width=\textwidth,
    lefthand width=\marginparwidth,
    halign upper=center,
  },
  toggle enlargement,
  grow to right by=\marginparsep+\marginparwidth,
  sidebyside gap=\marginparsep,
}}



\def\nilstrut{\rule{0sp}{0sp}}


\begin{document}

%==[ TITLE PAGE ]===============================================================

\thispagestyle{empty}
\noindent
\resizebox{\linewidth}{!}{\scshape koDi}\\[0.62em]
\resizebox{\linewidth}{!}{\scshape kommutative Diagramme für \TeX}\\[1.62em]
\resizebox{\linewidth}{!}{\scshape enchiridion}\par
\vfill
\marginpar{
  \resizebox{\linewidth}{!}{\scshape <VERSION>}\\
  \resizebox{\linewidth}{!}{\scshape \today}
}

%==[ FOREWORD ]=================================================================

\newpage
% \begin{adjustwidth}{.45\textwidth}{.45\textwidth-\marginparwidth-\marginparsep}
\noindent\koDi\ is a \TikZ\ library. Its purpose
is drawing commutative diagrams.
It is designed precisely to overcome
the shortcomings of traditional ones.
The syntax is minimalistic and intelligible.\par
\hfill{\itshape Paolo Brasolin}
% \end{adjustwidth}

\newpage
\section{Preliminaries}
\TikZ\ is the only dependency of \koDi.
This ensures compatibility with most \TeX\ flavours.
Furthermore, it can be invoked as a standalone as well as a \TikZ\ library.
Below are minimal working examples for the main dialects.

\begin{figure}[H]
  \begin{adjustwidth}{0sp}{-\marginparwidth-\marginparsep}
    \begin{subfigure}{\marginparwidth}
      \caption*{\TeX\ package}
      \begin{lstlisting}[gobble=8]

        \input kodi

        \kodi
          % diagram here
        \endkodi
        \bye
      \end{lstlisting}
    \end{subfigure}
    \hfill
    \begin{subfigure}{\marginparwidth}
      \caption*{\ConTeXt\ module}
      \begin{lstlisting}[gobble=8]

        \usemodule[kodi]
        \starttext
        \startkodi
          % diagram here
        \stopkodi
        \stoptext
      \end{lstlisting}
    \end{subfigure}
    \hfill
    \begin{subfigure}{\marginparwidth}
      \caption*{\LaTeX\ package}
      \begin{lstlisting}[gobble=8]
        \documentclass{article}
        \usepackage{kodi}
        \begin{document}
        \begin{kodi}
          % diagram here
        \end{kodi}
        \end{document}
      \end{lstlisting}
    \end{subfigure}
    \par
    \begin{subfigure}{\marginparwidth}
      \caption*{\TeX\ (\TikZ\ library)}
      \begin{lstlisting}[gobble=8]

        \input tikz
        \usetikzlibrary kodi

        \tikzpicture[kodi]
          % diagram here
        \endtikzpicture
        \bye
      \end{lstlisting}
    \end{subfigure}
    \hfill
    \begin{subfigure}{\marginparwidth}
      \caption*{\ConTeXt\ (\TikZ\ library)}
      \begin{lstlisting}[gobble=8]

        \usemodule[tikz]
        \usetikzlibrary[kodi]
        \starttext
        \starttikzpicture[kodi]
          % diagram here
        \stoptikzpicture
        \stoptext
      \end{lstlisting}
    \end{subfigure}
    \hfill
    \begin{subfigure}{\marginparwidth}
      \caption*{\LaTeX\ (\TikZ\ library)}
      \begin{lstlisting}[gobble=8]
        \documentclass{article}
        \usepackage{tikz}
        \usetikzlibrary{kodi}
        \begin{document}
        \begin{tikzpicture}[kodi]
          % diagram here
        \end{tikzpicture}
        \end{document}
      \end{lstlisting}
    \end{subfigure}
  \end{adjustwidth}
\end{figure}

A useful \TikZ\ feature exclusive to \LaTeX\ is externalization.
A small expedient is necessary to use it with \koDi.

\begin{marginfigure}[-2em]
  \caption*{\TikZ\ externalization}
  \begin{lstlisting}[gobble=4]
    \documentclass{article}
    \usepackage{tikz}
    \usetikzlibrary{kodi}
    \usetikzlibrary{external}
    \tikzexternalize
      [prefix=tikzpicfolder/]
    \begin{document}
    \begin{tikzpicture}[kodi]
      % diagram here
    \end{tikzpicture}
    \end{document}
  \end{lstlisting}
\end{marginfigure}

\newpage
\section{Quick tour}
Objects are typeset using the \lstinline!\obj! macro.
%
\begin{tcblisting}{kodi snippet}
\nilstrut\smash{\tikzpicture[baseline=(current bounding box.center),kodi]
\obj {X};
\endtikzpicture}
\end{tcblisting}
%
Almost every diagram is laid along a regular grid,
so the customary tabular syntax of \TeX\ is recognized.
%
\begin{tcblisting}{kodi snippet}
\nilstrut\smash{\tikzpicture[baseline=(current bounding box.center),kodi]
\obj {
  A & B \\
  C & D \\
};
\endtikzpicture}
\end{tcblisting}
%
\koDi\ objects are self-aware and clever enough to name themselves
so you can comfortably refer to them.
%
\begin{tcblisting}{kodi snippet}
\nilstrut\smash{\tikzpicture[baseline=(current bounding box.center),kodi]
\obj {\lim F};
\draw (lim F) circle (4ex);
\endtikzpicture}
\end{tcblisting}
%
Morphisms are typeset using the \lstinline!\mor! macro.
%
\begin{tcblisting}{kodi snippet}
\nilstrut\smash{\tikzpicture[baseline=(current bounding box.center),kodi]
\obj { A & B \\ };
\mor A f:-> B;
\endtikzpicture}
\end{tcblisting}
%
Commutative diagrams exist to study composition and commutation
so naturally \koDi\ allows for the chaining of morphisms and
the gluing of chains.
%
\begin{tcblisting}{kodi snippet}
\nilstrut\smash{\tikzpicture[baseline=(current bounding box.center),kodi]
\obj { A & B \\ C & D \\ };
\mor A -> B -> D;
\mor * -> C -> *;
\endtikzpicture}
\end{tcblisting}
%
These are the only two macros defined by \koDi.

There are more features so please continue reading if this got your attention.

\newpage
\section{Alternatives}
% http://texdoc.net/texmf-dist/doc/generic/xypic/xyrefer.pdf

This is \Xy. It is\ldots\ something.

\begin{tcblisting}{snippet}
\xymatrix{
 U \ar@/_/[ddr]_y \ar[dr] \ar@/^/[drr]^x \\
  & X \times_Z Y \ar[d]^q \ar[r]_p
                 & X \ar[d]_f            \\
  & Y \ar[r]^g & Z                       }
\end{tcblisting}

\null


\begin{tcblisting}{snippet}
\begin{tikzpicture}[kodi][/kD/every object/.append style={node distance=6cm}]
\obj { |(pb)| X \times_Z Y & X \\
                         Y & Z \\ };
\obj [above left=of pb] {U};
%
\mor[swap] pb p:-> X f:-> Z;
\mor        * q:-> Y g:-> *;
%
\mor                       U   -> pb;
\mor      :[bend left=10]  * x:-> X;
\mor[swap]:[bend right=10] * y:-> Y;
\end{tikzpicture}
\end{tcblisting}

\null


\begin{tcblisting}{snippet}
\psset{arrows=->, nodesep=3pt, linewidth=0.4pt, colsep=2em, rowsep=2em}
$\begin{psmatrix}
U \\
& X\times_Z Y & X \\
& Y & Z
\everypsbox{\scriptstyle}
\ncline{1,1}{2,2}
\ncarc[arcangle=-10]{1,1}{3,2}_{y}
\ncarc[arcangle=10]{1,1}{2,3}^{x}
\ncline{2,2}{3,2}>{q}
\ncline{2,2}{2,3}_{p}
\ncline{2,3}{3,3}<{f}
\ncline{3,2}{3,3}^{g}
\end{psmatrix}$
\end{tcblisting}



% \begin{tcblisting}{kodi snippet}
% %\nilstrut\smash{\tikzpicture[baseline=(current bounding box.center),kodi]
% \begin{tikzcd}
% U
% \arrow[drr, bend left, "x"]
% \arrow[ddr, bend right, "y"]
% \arrow[dr, dotted, "{(x,y)}" description] & & \\
% & X \times_Z Y \arrow[r, "p"] \arrow[d, "q"]
% & X \arrow[d, "f"] \\
% & Y \arrow[r, "g"]
% & Z
% \end{tikzcd}
% %}
% \end{tcblisting}


\newpage
\section{Syntax: objects}
The first of the two macros that \koDi\ offers is \lstinline|\obj|.
It is polymorphic and can draw both single objects and layouts.

\begin{lstlisting}
\obj@opt@ <object options> @/opt@{<math>};(@
  \marginpar{\scriptsize Orange denotes optional fragments.}@)
\obj@opt@ <layout options> @/opt@{<layout>};
\end{lstlisting}

Layouts are described using the customary \TeX\ tabular syntax.

\begin{lstlisting}
<layout>         := (@\itshape\underbar{<row> <row separator>}@)(@
  \marginpar{\scriptsize Underlined fragments are repeated one or more times.}@)
<row>            := <cell> (@\itshape\color{orange!80!black}\underbar{<cell separator> <cell>}@)
<row separator>  := \\ @opt@[<length>]@/opt@
<cell>           := @opt@|<object options>| @/opt@<math>
<cell separator> := & @opt@[<length>]@/opt@
\end{lstlisting}

The discretionary options syntax is analogous to standard \TikZ\ nodes and
matrices, respectively.

\begin{lstlisting}
<object options> := (@\itshape\color{orange!80!black}\underbar{[object keylist]}@) @opt@(<name>) at (<coordinate>)@/opt@
<layout options> := (@\itshape\color{orange!80!black}\underbar{[layout keylist]}@) @opt@(<name>) at (<coordinate>)@/opt@
\end{lstlisting}

\hfill$\therefore$\hfill\null

% \begin{tcblisting}{kodi snippet}
% \tikzpicture[kodi, overlay, remember picture]
% \obj (it) {A};
% \endtikzpicture
% \end{tcblisting}

% \begin{tcblisting}{kodi snippet}
% \tikzpicture[kodi, overlay, remember picture]
% \obj [below right] at (it) {B};
% \endtikzpicture
% \end{tcblisting}

% \begin{tcblisting}{kodi snippet, listing options={firstline=3,lastline=-1}}
% \tikzpicture[kodi,overlay,remember picture]
% \pgfkeys{/kD/every object/.append style={above right=.25in}}
% \obj [red] (kD) at (current page.south west) {\heartsuit kD};
% \endtikzpicture
% \end{tcblisting}

% \begin{tcblisting}{kodi snippet}
% \tikzpicture[kodi,overlay,remember picture]
% \draw (kD) circle (1.5em);
% \endtikzpicture
% \end{tcblisting}
% \null

\begin{tcblisting}{kodi snippet}
\SmashAndCenter{\tikzpicture[kodi, /kD/diagrams/square=2em]
\obj { A & B &[-1em] C \\
       D & E &       F \\[-1em]
       G & H &       I \\ };
\endtikzpicture}
\end{tcblisting}

Objects are automagically named; the latest homonymous prevails.

\begin{tcblisting}{kodi snippet}
\SmashAndCenter{\tikzpicture[kodi, /kD/diagrams/square=2.125em]
\obj { A & A \\ };
\draw (A) circle (1em);
\endtikzpicture}
\end{tcblisting}

Naming an object avoids its automatic labeling.

\begin{tcblisting}{kodi snippet}
\SmashAndCenter{\tikzpicture[kodi, /kD/diagrams/square=2.125em]
\obj { A & |(A')| A \\ };
\draw [red]   (A)  circle (1em);
\draw [green] (A') circle (1em);
\endtikzpicture}
\end{tcblisting}

Naming a layout lets you refer to objects by row and column.

\begin{tcblisting}{kodi snippet}
\SmashAndCenter{\tikzpicture[kodi, /kD/diagrams/square=2.125em]
\obj (M) { A & A \\ A & A \\ };
\draw [red]   (M-1-2) circle (1em);
\draw [green] (M-2-1) circle (1em);
\endtikzpicture}
\end{tcblisting}


\newpage
\section{Syntax: morphisms}
The second and last macro that \koDi\ offers is \lstinline|\mor|.
It can draw single or chained morphisms.

\begin{lstlisting}
\mor@opt@ <chain options> @/opt@<object>(@
  \itshape\underbar{\textvisiblespace<morphism>\textvisiblespace<object>}@);(@
  \marginpar{\scriptsize Whitespace marked as \textvisiblespace\ is mandatory.}@)
\end{lstlisting}

Source and target objects are referred to by their name.
  
\begin{lstlisting}
<object>   := @nws@(<name>)@/nws@ (@
  \marginpar{\scriptsize Blue fragments can be either enclosed in the shown delimiters, a \TeX\ group, or simply devoid of whitespace.}@)
\end{lstlisting}

Morphisms consist of one or more optional labels and an arrow.
  
\begin{lstlisting}
<morphism> := @opt@<labels> : @/opt@<arrow>
<labels>   := @nws@"<math>"@/nws@ XOR (@\underbar{[{\itshape "<math>", <label keylist>}]}@) (@
  \marginpar{\scriptsize Alternatives are separated by $\vert$s.}@)
<arrow>    := @nws@[<arrow keylist>]@/nws@
\end{lstlisting}
% <labels>   := @nws@"<math>"@/nws@ XOR @nws@[<label keylist>]@/nws@ XOR (@\underbar{[{\itshape <label keylist>}]}@)

Global options can be given to both labels and arrows.

\begin{lstlisting}
<chain options> := [<label keylist>] @opt@: [<arrow keylist>]@/opt@
\end{lstlisting}

\newpage
\section{Names}
As you'll have guessed by now, objects name themselves.

The process happens in three steps:
\begin{itemize}[nosep]
  \item expand tokens;
  \item replace characters;
  \item apply name, overwriting if necessary.
\end{itemize}

Each one can be configured in any \koDi\ scope with the keys.

While you're getting acquainted with the process
you can use the \lstinline|/kD/prompter| key to visibly
label any object with its generated name.

\begin{tcblisting}{snippet}
\begin{kodi}[prompter]
  \obj{
    A & \lim A \\
    \alpha\beta & \gamma \delta \\
  };
\end{kodi}
\end{tcblisting}

\newpage
\subsection{Shortcuts}
\begingroup\tcbset{trim/.default={3 and -1}}

Two special labels exist: {\ttfamily *} and {\ttfamily +}.

As a source, {\ttfamily *} evaluates to the head of the previous chain.

\begin{tcblisting}{snippet, trim}
\begin{codi}
\obj [square=2.5em] { A & B & C & \phantom{D} \\ };
\mor B -> C;
\mor * -> A;
\end{codi}
\end{tcblisting}

As a target, {\ttfamily *} evaluates to the tail of the previous chain.

\begin{tcblisting}{snippet, trim}
\begin{codi}
\obj [square=2.5em] { \phantom{A} & B & C & D \\ };
\mor B -> C;
\mor D -> *;
\end{codi}
\end{tcblisting}

The natural use case for {\ttfamily *} is chain gluing.

\begin{tcblisting}{snippet, trim}
\begin{codi}
\obj [square=2.5em] { A & B \\ D & C \\ };
\mor A -> B -> C;
\mor * -> D -> *;
\end{codi}
\end{tcblisting}

As a source, {\ttfamily +} evaluates to the tail of the previous chain.

\begin{tcblisting}{snippet, trim}
\begin{codi}
\obj [square=2.5em] { \phantom{A} & B & C & D \\ };
\mor B -> C;
\mor + -> D;
\end{codi}
\end{tcblisting}

As a target, {\ttfamily +} evaluates to the head of the previous chain.

\begin{tcblisting}{snippet, trim}
\begin{codi}
\obj [square=2.5em] { A & B & C & \phantom{D} \\ };
\mor B -> C;
\mor A -> +;
\end{codi}
\end{tcblisting}

The natural use case for {\ttfamily +} is chain extension.

\begin{tcblisting}{snippet, trim}
\begin{codi}
\obj [square=2.5em] { A & B & C & D \\ };
\mor B -> C;
\mor A -> + -> D;
\end{codi}
\end{tcblisting}

The meanings of {\ttfamily *} and {\ttfamily +} swap on opposite chains.

Chain extension can be obtained using {\ttfamily *}.

\begin{tcblisting}{snippet, trim}
\begin{codi}
\obj [square=2.5em] { A & B & C & D \\ };
\mor B <- C;
\mor D -> * -> A;
\end{codi}
\end{tcblisting}

Chain gluing can be obtained using {\ttfamily +}.

\begin{tcblisting}{snippet, trim}
\begin{codi}
\obj [square=2.5em] { A & B \\ D & C \\ };
\mor A <- B <- C;
\mor + -> D -> +;
\end{codi}
\end{tcblisting}

\endgroup

\newpage
\subsection{Expansion}
The expansion behaviour of the naming routine can be configured
inside any \CoDi\ scope using the \lstinline!expand! key.

\begin{lstlisting}
/codi/expand = none | once | full
\end{lstlisting}

The three available settings correspond to different degrees of expansion.
A side by side comparison completely illustrates their meanings.

\begin{tcblisting}{snippet, trim}
\begin{codi}
\def\B{Z}
\def\A{\B}
\obj{ |[expand=none]| \A &     % name: A (default)
      |[expand=once]| \A &     % name: B
      |[expand=full]| \A \\ }; % name: Z
\mor A -> B -> Z;
\end{codi}
\end{tcblisting}

\hfill$\therefore$\hfill\null

The default behaviour is to avoid expansion in compliance with the principle
that \emph{names should be predictable from the \emph{literal} code}.
Furthermore, it is seldom wise to liberally expand tokens.

There are circumstances in which it is useful to perform token expansion,
though. A useful application is procedural drawing.

\begin{tcblisting}{snippet, trim}
\begin{codi}
\foreach [count=\r] \l in {A,B,C}
  \foreach [count=\c] \n in {n-1,n,n+1}
    \obj [expand=full] at (3em*\c,-2em*\r) {\l_{\n}};
\mor (A_{n}) -> (B_{n+1}) -> (C_{n}) -> (B_{n-1}) -> (A_{n});
\end{codi}
\end{tcblisting}

In some cases finer control is needed. For instance, full expansion
yields unpractical results when parametrizing macros.

\begin{tcblisting}{snippet, trim}
\begin{codi}
\foreach [count=\c] \m in {\lim,\prod}
  \obj [expand=full] at (4em*\c,0) {\m F};
\mor (protect mathop {relax kern z@ mathgroup
                        symoperators lim}nmlimits@ F)
  -> (DOTSI prodop slimits@ F);
\end{codi}
\end{tcblisting}

This explains why a setting to force a single expansion exists.

\begin{tcblisting}{snippet, trim}
\begin{codi}
\foreach [count=\c] \m in {\lim,\prod}
  \obj [expand=once] at (4em*\c,0) {\m F};
\mor (lim F) -> (prod F);
\end{codi}
\end{tcblisting}

\newpage
\subsection{Replacement}
The character replacement behaviour of the naming routine can be configured
inside any \CoDi\ scope using various keys.

\begin{lstlisting}
/codi/replace character = <character> with <character>
/codi/replace charcode = <charcode> with <character>
/codi/remove characters = <characters>
/codi/remove character = <character>
/codi/remove charcode = <charcode>
\end{lstlisting}

You can set up a replacement for any character, using the character code for
the hardest to type, like {\ttfamily \textvisiblespace} or {\ttfamily \textbackslash}.

\begin{tcblisting}{snippet, trim}
\begin{codi}[tetragonal]
\obj{ |[replace character=F with G]| \lim F &     % name: lim G
      |[remove character=F]|         \lim F \\    % name: lim
      |[replace charcode=92 with /]| \lim F &     % name: /lim F
      |[remove charcode=32]|         \lim F \\ }; % name: limF
\mor (lim G) -> (lim) -> (/lim F) -> (limF);
\end{codi}
\end{tcblisting}

\hfill$\therefore$\hfill\null

The default behaviour is removal of the minimal set of universally annoying%
\footnote{The difficult part is not creating the names but having to type them.}
characters: {\ttfamily (),.:}  have special meanings to \TikZ\ while
{\ttfamily \textbackslash} is impossible to type by ordinary means, so they're \emph{kaput}.

Each one can be restored by replacing it with itself. Don't.

Another egregiously bad idea is replacing characters with spaces.
It's tempting because it solves a somewhat common edge case.

\begin{tcblisting}{snippet, trim}
\begin{codi}[tetragonal]
\obj{ \beta & F & b\eta \\ };
\mor F -> beta;
\end{codi}
\end{tcblisting}

Since characters in names are literal, this causes whitespace
duplication and names become inaccessible by ordinary means.

\begin{tcblisting}{snippet, trim}
\begin{codi}[tetragonal]
\obj [replace charcode=92 with \space]
  { \beta & b\eta & \beta \eta \\ };
\mor beta -> (b eta) -> (beta \space eta);
\end{codi}
\end{tcblisting}

The wise solution is writing better code.

\begin{tcblisting}{snippet, trim}
\begin{codi}[tetragonal]
\obj{ \beta & F & b \eta \\ };
\mor F -> beta;
\end{codi}
\end{tcblisting}


\newpage
\subsection{Overwriting}
The name overwriting behaviour of the naming routine can be configured
inside any \koDi\ scope using the \lstinline!overwrite! key.

\begin{lstlisting}
/kD/overwrite = false | alias | true
\end{lstlisting}

The three available settings correspond to different naming priorities.
A side by side comparison completely illustrates their meanings.

\begin{tcblisting}{snippet, trim}
\begin{kodi}
\obj{ |[overwrite=false] (A')| A &     % names: A'    (default)
      |[overwrite=alias] (B')| B &     % names: B', B
      |[overwrite=true]  (C')| C \\ }; % names:     C
\mor A' -> B';
\mor B  -> C;
\end{kodi}
\end{tcblisting}

\hfill$\therefore$\hfill\null

The default behaviour avoids overwriting explicit labels in order
to give you a simple means of naming conflict resolution.

\begin{tcblisting}{snippet, trim}
\begin{kodi}[golden]
\obj {        A & |(A')| A \\
       |(Z')| Z &        Z \\ };
\mor A -> A';
\mor Z -> Z';
\end{kodi}
\end{tcblisting}

Sometimes you might want an object to have both a literal and a
semantic alias.

\begin{tcblisting}{snippet, trim}
\begin{kodi}
\obj [overwrite=alias] { A & |(center)| B & |(right)| C \\ };
\mor A -> B;
\mor center -> right;
\end{kodi}
\end{tcblisting}

The hard overwriting behaviour ignores any label except generated
ones; it exists for completeness and debugging purposes.

\newpage
\section{Gallery}
\begin{tikzpicture}[kodi]
\obj[golden] { & \ker a   & \ker b   & \ker c   &   \\
               & A        & B        & C        & 0 \\
      |(0')| 0 & A'       & B'       & C'       &   \\
               & \coker a & \coker b & \coker c &   \\ };
% horizontal chains
\mor   (ker a) ->   (ker b) ->   (ker c);
\mor (coker a) -> (coker b) -> (coker c);
\mor       A  f :-> B  g :-> C -> 0;
\mor 0' -> A' f':-> B' g':-> C';
% vertical chains
\mor[near start] (ker a) -> A a:-> A' -> (coker a);
\mor[near start] (ker b) -> B b:-> B' -> (coker b);
\mor[near start] (ker c) -> C c:-> C' -> (coker c);
% the snake
\coordinate (tail) at ($(ker b)!9/4!(ker c)$);
\coordinate (head) at ($(coker b)!9/4!(coker a)$);
\coordinate (belly) at ($(B)!9/16!(B')$);
\draw[/kD/arrows/crossing over, ->, rounded corners]
  (ker c) -- (tail) -- (tail|-belly)
          -- (belly-|head) -- (head) -- (coker a);

\end{tikzpicture}

\begin{tikzpicture}[kodi]
% \usepackage{newunicodechar}
% \newunicodechar{ı}{\mathbf 1}
% \newunicodechar{×}{\otimes}

\foreach [count=\n] \o in
    {((w×x)×y)×x,
     (w×(x×y))×x,
     w×((x×y)×x),
     w×(x×(y×x)),
     (w×x)×(y×x)}
  \obj (\n) at ({72*\n:7em}) {\o};

\mor 1 "a_{w,x,y}×ı_z": -> 2
         "a_{w,x×y,z}": -> 3
       "ı_w×a_{x,y,z}": -> 4;
\mor *   "a_{w×x,y,z}": -> 5
         "a_{w,x,y×z}": -> *;

\end{tikzpicture}

\begin{kodi}
  \obj[comb]{ |(P)| A \times_Z B & B \\
                    A            & Z \\ };
  \obj at ($(Z)!2!(P)$) {Q};

  \mor[swap] P p_1:-> A f:-> Z;
  \mor       * p_2:-> B g:-> *;

  \mor[swap]:[bend right] Q q_1:-> A;
  \mor      :[bend left]  * q_2:-> B;
  \mor [mid]:[dashed]     *   u:-> P;
\end{kodi}

\begin{kodi}
  \pgfkeys{/katharizo/expand=full}

  \foreach [count=\m] \a in {A,B,C}
    \foreach [count=\n] \i in {n-2,n-1,n,n+1,n+2}
      \obj at ({5em*\n,-3em*\m}) {\a_{\i}};

  \mor (C_{n+1}) -> (B_{n-1});
\end{kodi}


\begin{kodi}
  \obj{
    H_{q+2}(X)   &  H_{q+2}(X,Y) & H_{q+1}(Y,Z) & H_{q  }(Z)   \\
    H_{q+2}(Y)   &  H_{q+2}(X,Z) & H_{q+1}(Y)   & H_{q+1}(X,Z) \\
    H_{q+2}(Y,Z) &  H_{q+1}(Z) & H_{q+1}(X) &  \\
  };
\end{kodi}

\begin{kodi}
  \obj{        & |(A)|  \vdots  & |(B)|  \vdots  & |(C)|  \vdots  &               \\
  |(0{n+1})| 0 &        A_{n+1} &        B_{n+1} &        C_{n+1} & |(0'{n+1})| 0 \\
  |(0{n})  | 0 &        A_{n}   &        B_{n}   &        C_{n}   & |(0'{n})  | 0 \\
  |(0{n-1})| 0 &        A_{n-1} &        B_{n-1} &        C_{n-1} & |(0'{n-1})| 0 \\
               & |(A')| \vdots  & |(B')| \vdots  & |(C')| \vdots  &               \\};

  \foreach \n in {n+1,n,n-1}
    \mor (0{\n}) -> (A_{\n}) "\alpha_{\n}":-> (B_{\n})
                              "\beta_{\n}":-> (C_{\n}) -> (0'{\n});

  \foreach \l/\q in {A/,B/',C/''}
    \mor (\l) -> (\l_{n+1}) "\partial\q_{n+1}":-> (\l_{n})
                            "\partial\q_{n}"  :-> (\l_{n-1}) -> (\l');
\end{kodi}

\begin{kodi}
\obj[square]{A&B&C&D\\E&F&G&H\\I&L&M&N\\};

\mor:                     B -> C;

% * src = prv first src
\mor:[draw=none]          B -> C;
\mor:[bend right, green]  * -> D;

% * tar = prv last tar
\mor:[draw=none]          B -> C;
\mor:[bend right, blue]   A -> *;

\mor:[draw=none]          B -> C;
\mor:[bend left, red]     A -> * -> D;

\mor:                     F -> G;

% + src = prv last tar
\mor:[draw=none]          F -> G;
\mor:[bend right, green]  + -> H;

% + tar = prv first src
\mor:[draw=none]          F -> G;
\mor:[bend right, blue]   E -> +;

\mor:[draw=none]          F -> G;
\mor:[bend left, red]     E -> + -> H;

% so, basically, * is the opposite of +
\mor:      M -> L;
\mor:[red] I <- * <- N;
\end{kodi}

% \clearpage

\null\vfill

\marginpar{\begin{smashAndCenter}\begin{kodi}[comb=base 3.2em angle 60]

  \clip (-1.5,-1.5) rectangle (3.5,2.5);
      
  \foreach \i/\j in {-2/-2,0/-2,2/-2,-3/-1,-1/-1,1/-1,3/-1,-2/0,0/0,2/0,-3/1,-1/1,1/1,3/1,-2/2,0/2,2/2}
    \fill [gray!15] (\i,\j) -- +(1,1) -- +(2,0);
  
  \foreach \i in {-1, ..., 3} \foreach \j in {-1, ..., 2} {
    \pgfmathsetmacro\dotsize{isodd(int(abs(\i)+abs(\j)))?1:2}
    \fill [gray] (\i,\j) circle (\dotsize pt);
  }

  \draw [red, dashed, ultra thick] (0,0) -- (2,0);
  \draw [red, decorate,decoration={brace,amplitude=4pt,mirror},yshift=-4pt]
  (0,0) -- (2,0) node [below, midway, font=\tiny]  {base};
  \draw [ultra thick, green, dashed]
  (.75,.75) arc (60:30:2.4em) coordinate (wot) arc (30:0:2.4em);
  \path [green, late options={name=wot,pin={[pin edge={green,thin,-}, inner sep=0, font=\tiny]30:angle}}];
  \draw [thick, blue, ->, >=stealth] (0,0) -- (1,0);
  \draw [thick, blue, ->, >=stealth] (0,0) -- (0,1);
\end{kodi}\end{smashAndCenter}}

\vfill

\marginpar{\begin{smashAndCenter}\begin{kodi}[rectangular={2em}{2.5em}]

  \clip (-1.5,-1.5) rectangle (2.5,2.5);
      
  \foreach \i in {-2, ..., 2} \foreach \j in {-2, ..., 2} {
    \pgfmathparse{int(1+abs(\i)+abs(\j))}
    \ifodd\pgfmathresult
      \fill [gray!15] (\i,\j) rectangle +(1,1);
    \fi
  }
  
  \foreach \i in {-1, ..., 2} \foreach \j in {-1, ..., 2}
    \fill [gray] (\i,\j) circle (2pt);

  \draw [red, dashed, ultra thick] (0,0) -- (1,0);
  \draw [red, decorate,decoration={brace,amplitude=4pt,mirror},yshift=-4pt]
  (0,0) -- (1,0) node [below, midway, font=\tiny]  {base};
  
  \draw [green, dashed, ultra thick] (0,0) -- (0,1);
  \draw [green, decorate,decoration={brace,amplitude=4pt},xshift=-4pt]
  (0,0) -- (0,1) node [left, midway, font=\tiny, inner sep=.5em]  {height};
  
  \draw [thick, blue, ->, >=stealth] (0,0) -- (1,0);
  \draw [thick, blue, ->, >=stealth] (0,0) -- (0,1);
\end{kodi}\end{smashAndCenter}}

\vfill

% \begin{kodi}[golden=3em]
%   \fill [gray, opacity=.5] (0.309,0) -- (0.618,1) arc (63.43:0:1.31*1.6em) -- cycle;
%   \fill [gray, opacity=.5] (0,0) -- (0,1) arc (90:0:0.62*3em) -- cycle;
%   \draw [dashed] rectangle (0.618,1);

  
%   \foreach \i in {-1, ..., 2} \foreach \j in {-1, ..., 2} {
%     \fill [gray] (\i,\j) circle (2 pt);
%   }
  
%   \draw [red, dashed, ultra thick] (0,0) -- (1,0);
%   \draw [red, decorate,decoration={brace,amplitude=4pt,mirror},yshift=-4pt]
%   (0,0) -- (1,0) node [below, midway, font=\tiny]  {base};
  
%   \draw [green, dashed, ultra thick] (0,0) -- (0,1);
%   \draw [green, decorate,decoration={brace,amplitude=4pt},xshift=-4pt]
%   (0,0) -- (0,1) node [left, midway, font=\tiny]  {height};
  
%   \draw [thick, blue, ->, >=stealth] (0,0) -- (1,0);
%   \draw [thick, blue, ->, >=stealth] (0,0) -- (0,1);
  
% \end{kodi}


\end{document}
