Objects are typeset using the \lstinline|\obj| macro.

\begin{tcblisting}{snippet, trim}
\begin{kodi}
\obj {X};
\end{kodi}
\end{tcblisting}

Almost every diagram is laid along a regular grid,
so the customary tabular syntax of \TeX\ is recognized.

\begin{tcblisting}{snippet, trim}
\begin{kodi}
\obj {
  A & B \\
  C & D \\
};
\end{kodi}
\end{tcblisting}

\koDi\ objects are self-aware and clever enough to name themselves
so you can comfortably refer to them.

\begin{tcblisting}{snippet, trim}
\begin{kodi}
\obj {\lim F};
\draw (lim F) circle (4ex);
\end{kodi}
\end{tcblisting}

Morphisms are typeset using the \lstinline!\mor! macro.

\begin{tcblisting}{snippet, trim}
\begin{kodi}
\obj { A & B \\ };
\mor A f:-> B;
\end{kodi}
\end{tcblisting}

Commutative diagrams exist to illustrate composition and commutation,
so \koDi\ allows arrow chaining and chain gluing.

\begin{tcblisting}{snippet, trim}
\begin{kodi}
\obj { A & B \\ C & D \\ };
\mor A -> B -> D;
\mor * -> C -> *;
\end{kodi}
\end{tcblisting}

These are the only two macros defined by \koDi.

There are more features, though.\\
Read on if this caught your attention.
