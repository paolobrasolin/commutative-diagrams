The character replacement behaviour of the naming routine can be configured
inside any \koDi\ scope using various keys.

\begin{lstlisting}
/kD/replace character = <character> with <character>
/kD/replace charcode = <charcode> with <character>
/kD/remove character = <character>
/kD/remove charcode = <charcode>
\end{lstlisting}




You can set up a replacement for any character, using the character code for
the hardest to type, like {\ttfamily \textvisiblespace} or {\ttfamily \textbackslash}.
The default behaviour removes {\ttfamily \$(),.:\textbackslash} since they
are generally dangerous in \TikZ.

\begin{tcblisting}{kodi snippet}
\SmashAndCenter{\begin{kodi}\pgfkeys{/kD/diagrams/golden}
\obj{ |[replace character=F with G]| \lim F &     % name: lim G
      |[remove character=F]|         \lim F \\    % name: lim
      |[replace charcode=92 with /]| \lim F &     % name: /lim F
      |[remove charcode=32]|         \lim F \\ }; % name: limF
\mor (lim G) -> (lim) -> (/lim F) -> (limF);
\end{kodi}}
\end{tcblisting}

\hfill$\therefore$\hfill\null

Wooot.

\begin{tcblisting}{kodi snippet}
\SmashAndCenter{\begin{kodi}\pgfkeys{/kD/diagrams/golden}
\obj{ |[replace character=F with G]| \lim F &     % name: lim G
      |[remove character=F]|         \lim F \\    % name: lim
      |[replace charcode=92 with \space]| x\lim F &     % name: /lim F
      |[remove charcode=32]|         \lim F \\ }; % name: limF
\mor (lim G) -> (lim) -> (x lim F) -> (limF);
\end{kodi}}
\end{tcblisting}

