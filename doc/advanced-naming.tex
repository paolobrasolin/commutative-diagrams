As you'll have guessed by now, objects name themselves.

The process happens in three steps:
\begin{itemize}[nosep]
  \item expand tokens;
  \item replace characters;
  \item apply name, overwriting if necessary.
  \end{itemize}

Each one can be configured in any \koDi\ scope with the keys.

You can control the degree of expansion; the default behaviour (no expansion)
shields you from problems with unsafe macros.

\begin{tcblisting}{kodi snippet}
\SmashAndCenter{\begin{kodi}
\def\B{Z} \def\A{\B}
\obj{ |[expand=none]| \A &     % name: A (default)
      |[expand=once]| \A &     % name: B
      |[expand=full]| \A \\ }; % name: Z
\mor A -> B -> Z;
\end{kodi}}
\end{tcblisting}

You can set up a replacement for any character, using the character code for
the hardest to type, like {\ttfamily \textvisiblespace} or {\ttfamily \textbackslash}.
The default behaviour removes {\ttfamily \$(),.:\textbackslash} since they
are generally dangerous in \TikZ.

\begin{tcblisting}{kodi snippet}
\SmashAndCenter{\begin{kodi}\pgfkeys{/kD/diagrams/golden}
\obj{ |[replace character=F with G]| \lim F &     % name: lim G
      |[remove character=F]|         \lim F \\    % name: lim
      |[replace charcode=92 with /]| \lim F &     % name: /lim F
      |[remove charcode=32]|         \lim F \\ }; % name: limF
\mor (lim G) -> (lim) -> (/lim F) -> (limF);
\end{kodi}}
\end{tcblisting}

Overwriting behaviour can also be configured. The default setting is the ideal
for manually solving automatic names conflicts.

\begin{tcblisting}{kodi snippet}
\SmashAndCenter{\begin{kodi}
\obj{ |[overwrite=false] (A')| A &     % names: A'    (default)
      |[overwrite=alias] (B')| B &     % names: B', B
      |[overwrite=true]  (C')| C \\ }; % names:     C
\mor A' -> B';
\mor B  -> C;
\end{kodi}}
\end{tcblisting}

\hfill$\therefore$\hfill\null

Expansion control is mainly useful for procedural drawing.

\begin{tcblisting}{kodi snippet}
\SmashAndCenter{\begin{kodi}
\foreach [count=\r] \l in {A,B,C}
  \foreach [count=\c] \n in {n-1,n,n+1}
    \obj [expand=full] at (3em*\c,-1.6em*\r) {\l_{\n}};
\mor (A_{n}) -> (B_{n+1}) -> (C_{n}) -> (B_{n-1}) -> (A_{n});
\end{kodi}}
\end{tcblisting}

