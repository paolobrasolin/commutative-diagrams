The character replacement behaviour of the naming routine can be configured
inside any \CoDi\ scope using various keys.

\begin{lstlisting}
/codi/replace character = <character> with <character>
/codi/replace charcode = <charcode> with <character>
/codi/remove characters = <characters>
/codi/remove character = <character>
/codi/remove charcode = <charcode>
\end{lstlisting}

You can set up a replacement for any character, using the character code for
the hardest to type, like {\ttfamily \textvisiblespace} or {\ttfamily \textbackslash}.

\begin{tcblisting}{snippet, trim}
\begin{codi}[tetragonal]
\obj{ |[replace character=F with G]| \lim F &     % name: lim G
      |[remove character=F]|         \lim F \\    % name: lim
      |[replace charcode=92 with /]| \lim F &     % name: /lim F
      |[remove charcode=32]|         \lim F \\ }; % name: limF
\mor (lim G) -> (lim) -> (/lim F) -> (limF);
\end{codi}
\end{tcblisting}

\hfill$\therefore$\hfill\null

The default behaviour is removal of the minimal set of universally annoying%
\footnote{The difficult part is not creating the names but having to type them.}
characters: {\ttfamily (),.:}  have special meanings to \TikZ\ while
{\ttfamily \textbackslash} is impossible to type by ordinary means, so they're \emph{kaput}.

Each one can be restored by replacing it with itself. Don't.

Another egregiously bad idea is replacing characters with spaces.
It's tempting because it solves a somewhat common edge case.

\begin{tcblisting}{snippet, trim}
\begin{codi}[tetragonal]
\obj{ \beta & F & b\eta \\ };
\mor F -> beta;
\end{codi}
\end{tcblisting}

Since characters in names are literal, this causes whitespace
duplication and names become inaccessible by ordinary means.

\begin{tcblisting}{snippet, trim}
\begin{codi}[tetragonal]
\obj [replace charcode=92 with \space]
  { \beta & b\eta & \beta \eta \\ };
\mor beta -> (b eta) -> (beta \space eta);
\end{codi}
\end{tcblisting}

The wise solution is writing better code.

\begin{tcblisting}{snippet, trim}
\begin{codi}[tetragonal]
\obj{ \beta & F & b \eta \\ };
\mor F -> beta;
\end{codi}
\end{tcblisting}

