The name overwriting behaviour of the naming routine can be configured
inside any \CoDi\ scope using the \lstinline!overwrite! key.

\begin{lstlisting}
/codi/overwrite = false | alias | true
\end{lstlisting}

The three available settings correspond to different naming priorities.
A side by side comparison completely illustrates their meanings.

\begin{tcblisting}{snippet, trim}
\begin{codi}
\obj{ |[overwrite=false] (A')| A &     % names: A'    (default)
      |[overwrite=alias] (B')| B &     % names: B', B
      |[overwrite=true]  (C')| C \\ }; % names:     C
\mor A' -> B';
\mor B  -> C;
\end{codi}
\end{tcblisting}

\hfill$\therefore$\hfill\null

The default behaviour avoids overwriting explicit labels in order
to give you a simple means of naming conflict resolution.

\begin{tcblisting}{snippet, trim}
\begin{codi}[tetragonal]
\obj {        A & |(A')| A \\
       |(Z')| Z &        Z \\ };
\mor A -> A';
\mor Z -> Z';
\end{codi}
\end{tcblisting}

Sometimes you might want an object to have both a literal and a
semantic alias.

\begin{tcblisting}{snippet, trim}
\begin{codi}
\obj [overwrite=alias] { A & |(center)| B & |(right)| C \\ };
\mor A -> B;
\mor center -> right;
\end{codi}
\end{tcblisting}

The hard overwriting behaviour ignores any label except generated
ones; it exists for completeness and debugging purposes.
