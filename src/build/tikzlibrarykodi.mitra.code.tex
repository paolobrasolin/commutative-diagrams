%% tikzlibrarykodi.mitra.code.tex
%% Copyright 2017 Paolo Brasolin
%
% This work may be distributed and/or modified under the
% conditions of the LaTeX Project Public License, either version 1.3
% of this license or (at your option) any later version.
% The latest version of this license is in
%   <http://www.latex-project.org/lppl.txt>
% and version 1.3 or later is part of all distributions of LaTeX
% version 2005/12/01 or later.
%
% This work has the LPPL maintenance status `maintained'.
%
% The Current Maintainer of this work is Paolo Brasolin
% who can be contacted at <paolo.brasolin@gmail.com>.
%
% This work consists of the following files:

% μήτρα • (mítra)
%   1. womb
%   2. matrix
%   3. mold

% Mitra is an alternative parsing mechanism for TiKz matrices of nodes.
% It couples with ozos to ensure the node contents  always pass through the TikZ key.

\usetikzlibrary{matrix}
\usetikzlibrary{kodi.koinos}

%==[ TikZ/pgf layer ]===========================================================

\pgfkeys{
  /mitra/every matrix/.style={
    /tikz/matrix,
    /tikz/cells={
      /tikz/anchor=base
    }
  },
  /mitra/every node/.style={},
  % NOTE: the alisa key *cannot* be put along with global node styles
  %   because order of execution.
  /mitra/matrix coordinates alias/.style={
    /tikz/alias=\tikzmatrixname-\the\pgfmatrixcurrentrow-\the\pgfmatrixcurrentcolumn,
  }
}

%==[ main macro ]===============================================================

\def\kDMitra
  {\kDMitraFetchMatrixThen
  {\kDMitraParseMatrixTableThen
   \kDMitraOutput}}

% General purpose scratch register.
\newtoks\kDMitraTmpTok

%==[ fetching routine ]=========================================================

% I use the general one implemented by koinos.
\let\kDMitraFetchMatrixThen\kDFetchOptAndGrpThen

%==[ parsing routine: /table ]==================================================

\newtoks\kDMitraMatOutTok

\def\kDMitraParseMatrixTableThen#1{%
  \kDMitraMatOutTok={}%
  \expandafter\kDMitraParseTable\the\kDGrpTok\kD
  #1}

\def\kDMitraParseTable
  {\kDMitraParseAllRowsThen\relax}

\def\kDMitraParseAllRowsThen#1{%
  \def\kDExit{#1}%
  \kDIfNextHardCh\kD
    {\expandafter\kDExit\kDGobbleHardTok}%
    {\kDMitraParseOneRowThen{\kDMitraParseAllRowsThen{#1}}}}

%==[ parsing routine: /table/row ]==============================================

\def\kDMitraParseOneRowThen#1%
  {\kDMitraMarkRowEndBefore
  {\kDMitraParseAllColsThen
  {\kDMitraParseRowEndThen
  {#1}}}}

\begingroup\catcode`&=4 % ConTeXt hack
\gdef\kDMitraMarkRowEndBefore#1#2\\{#1#2&\kD\\}
\endgroup

\def\kDMitraParseAllColsThen#1{%
  \def\kDExit{#1}%
  \kDIfNextHardCh\kD
    {\expandafter\kDExit\kDGobbleHardTok}%
    {\kDMitraParseOneColThen{\kDMitraParseAllColsThen{#1}}}}

\def\kDMitraParseRowEndThen#1%
  {\kDMitraFetchRowEndThen
  {\kDMitraPrintRowEndThen
  {#1}}}

\def\kDMitraFetchRowEndThen#1\\{\kDMitraMaybeFetchRowOptionsThen{#1}}

\newtoks\kDMitraRowOptTok

\def\kDMitraPrintRowEndThen#1{%
  \edef\kDAct{\noexpand\kDMitraTmpTok={\noexpand\\\the\kDMitraRowOptTok}}%
  \kDAct
  \kDAppend\kDMitraTmpTok\kDMitraMatOutTok
  #1}

\def\kDMitraMaybeFetchRowOptionsThen#1{%
  \kDMitraRowOptTok={}%
  \kDIfNextHardCh[%
    {\kDMitraFetchRowOptionsThen{#1}}%
    {#1}}

\def\kDMitraFetchRowOptionsThen#1[#2]{\kDMitraRowOptTok={[#2]}#1}

%==[ parsing routine: /table/row/column ]=======================================

\def\kDMitraParseOneColThen#1%
  {\kDMitraParseCellThen
  {\kDMitraParseColEndThen
  {#1}}}

\def\kDMitraParseColEndThen#1%
  {\kDMitraMaybeFetchColOptionsThen
  {\kDMitraMaybePrintColEndThen
  {#1}}}

\newtoks\kDMitraColOptTok

\def\kDMitraMaybeFetchColOptionsThen#1{%
  \kDMitraColOptTok={}%
  \kDIfNextHardCh[%
    {\kDMitraFetchColOptionsThen{#1}}%
    {#1}}

\def\kDMitraFetchColOptionsThen#1[#2]{\kDMitraColOptTok={[#2]}#1}

% TODO: make catcode change temporary with wrapping group and global defs
%   making sure auxiliary macros aren't botched (and don't botch anything).
\catcode`&=\active
\def\kDMitraMaybePrintColEndThen#1{%
  \kDIfNextHardCh\kD
    {#1}%
    {%
      \edef\kDAct{\noexpand\kDMitraTmpTok={\noexpand&\the\kDMitraColOptTok}}%
      \kDAct
      \kDAppend\kDMitraTmpTok\kDMitraMatOutTok
      #1%
    }}
\catcode`&=4

%==[ parsing routine: /table/row/column/cell ]==================================

\def\kDMitraParseCellThen#1%
  {\kDMitraMaybeFetchCellOptionsThen
  {\kDMitraFetchCellContentThen
  {\kDMitraPrintCellThen
  {#1}}}}

\newtoks\kDMitraCelOptTok

\def\kDMitraMaybeFetchCellOptionsThen#1{%
  \kDMitraCelOptTok={}%
  \kDIfNextHardCh|%
    {\kDMitraFetchCellOptionsThen{#1}}
    {#1}}

\def\kDMitraFetchCellOptionsThen#1|#2|{\kDMitraCelOptTok={#2}#1}

\newtoks\kDMitraCelCntTok

\def\kDMitraFetchCellContentThen#1#2&{%
  \kDMitraCelCntTok={#2}%
  \kDTrimLeadingSpace\kDMitraCelCntTok
  \kDTrimTrailingSpace\kDMitraCelCntTok
  #1}

\def\kDMitraMaybeDoCellThen#1{%
  \kDIfNextHardCh\kD
    {#1}%
    {\kDMitraParseOneColThen{#1}}}

\def\kDMitraPrintCellThen#1{%
  \edef\kDAct{%
    \noexpand\kDMitraTmpTok={%
      \noexpand\node
        [/mitra/every node]%
        \the\kDMitraCelOptTok
        [node contents={\the\kDMitraCelCntTok}]%
        [/mitra/matrix coordinates alias];%
      \kDMitraMaybeDumpCell}}%
  \kDAct
  \kDAppend\kDMitraTmpTok\kDMitraMatOutTok
  #1}

%==[ dumping routine ]==========================================================

% TODO: revise the mechanism
\def\kDMitraMaybeDumpCell{\noexpand\pgfextra{%
  \noexpand\kDDump{'\noexpand\the\noexpand\pgfmatrixcurrentrow-\noexpand\the\noexpand\pgfmatrixcurrentcolumn':}%
  \noexpand\kDDump{\space\space options: '\the\kDMitraCelOptTok'}%
  \noexpand\kDDump{\space\space content: '\the\kDMitraCelCntTok'}}}

%==[ output routine ]===========================================================

\def\kDMitraOutput{%
  \edef\kDAct{%
    \noexpand\kDMitraTmpTok={%
      \noexpand\matrix
        [/mitra/every matrix]%
        \the\kDOptTok
        {\the\kDMitraMatOutTok};}}%
  \kDAct
  \the\kDMitraTmpTok}

%==[ notes ]====================================================================

% \newif\ifConTeXt
% Trick stolen from iftex. The second line is expanded inside the group so
% the global scope isn't polluted by \csname defining the token.
% \begingroup\expandafter\expandafter\expandafter\endgroup\expandafter
%   \ifx\csname starttext\endcsname\relax\ConTeXtfalse\else\ConTeXttrue\fi
% \ifConTeXt\catcode`|=12 \fi% TODO: why was the space necessary again?
